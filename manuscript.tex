% plagiarism check
% cohesion check
% grammar check (firt two page of new paper)
\documentclass[man]{apa6}
\title{Robots, Are They Overlords or Serf-Laborers?}
\author{Murat Ambarkutuk}
\affiliation{English Language Institute, University of Delaware \\ murata@udel.edu}
\note{Manuscript, \today \\ \LaTeX}
\usepackage{apacite}
% modify it!
\abstract{Artificial Intelligence (AI) has been one of the most debated issues in the field of engineering.
Robots are also subject to these debates because that tangible spin-off technology that will be run by such an AI that will have superior cognitive and perceptive abilities than that of humankind. 
Many researchers argue that the creation of AI will deeply elevate society and it will help humankind to solve major societal problems, while others refute the arguments for that conceptual invention.
This paper explores these issues that the creation of AI will raise, and it attributes these problems to widely varying definitions and the applications of these technologies.
}
\keywords{artificial intelligence, robotics, ethics, society} 
\shorttitle{Overlords vs. Serf-Laborers}
\leftheader{Even-Numbered Page Header} %Change it
\begin{document}
\maketitle

\section{Introduction}
Even though the term robot was coined two centuries ago and preliminary contemplation of Artificial Intelligence (AI) emerged decades ago, some of the problems that have been raised by these two terms are yet to be solved.
Sometimes, solutions for any arbitrary problem can be addressed by seeing the bigger picture rather than scrutinizing tiny details.
Thus, this paper explores problems related to legal, ethical frameworks and then analyzes the numerous definitions and applications of these technologies which pose challenges on varying levels, from individual to societal.
The organization of this paper is as follows: The first and second sections scrutinize the definitions and the applications of robots, respectively, to attribute the reasons for these issues, and the third section then categorizes the issues that robotics have brought about.

\section{What is a Robot?}
%One common definition of AI is cited by \citeA{bost}, "A super-intelligence is any intellect that vastly outperforms the best human brains in practically every field, including scientific creativity, general %wisdom and social skills."
%This definition clearly distinguishes specific domain intelligent agents and AI.
% Based on this definition a computer system, Deep Blue, which is a well-known computer system for its success in chess can be considered smart within a narrow domain rather than a genuine AI supported by limitations that \citeauthor{duff} \citeyear[p.~34]{duff} speculates.
\par

By literary definition the term robot is derived from the Czech word "roboti" meaning serf-laborer. 
However, this definition closely referring to a slave-like laborer has become something brand new as technology advances.
In his article entitled "The Sheer Difficulty of Defining What a Robot Is", \citeA{pear} discusses two distinctive definitions of robotics that are posed by the prominent experts in the field of robotics.
These two definitions emphasize different aspects of robotics creating important distinctions among them.
The first definition puts emphasis on the successful combination of a rigid body responsible for interacting with its environment and a central processing unit responsible for reasoning tasks based on perception and cognition.
A robot cannot accomplish its tasks without a software and hardware collaboration, as \citeA{pear} cites in his article.
The second definition, however, takes autonomy into account.
A robot, \citeA{pear} quotes in his article, must have a detailed scheme of any action formed by a rigorous plan, accurate reasoning and precise action to interact with its physical environment, which closely aligns with the definition provided by \citeauthor{lin} \citeyear{lin}. 
\citeauthor{lin} \citeyear[p.~943]{lin} define a robot as "an engineered machine that senses, thinks, and acts".
In other words, roboticists have not been able  to reach a consensus on the definition of robots, the embodiment of AI, \citeA{lin} note.
\par

\section{What (should / does) a Robot do?}
% A strip from the old paper
In order to make its obscure definition clear, the tasks of AI should be addressed.
On the one hand,  \citeA[p.~943]{lin} note that robots are employed in tasks called "three D's" referring to "dull, dirty, or dangerous" tasks.
Based on these tasks, a robot should replace a human where the task contains either mundane chores, or risks of contamination, or threats to human lives.
\par 
%The task of AI is to create methods to enable machines to obtain symbolic data sets and manipulate them in order to solve generic problems in the most energy efficient and fastest fashion.
%The theory suggests that it is possible to achieve by imitating the processes of logical reasoning and decision making.
%Consequently, cognitive and perceptual systems of the human body have been scrutinized by researchers in order to emulate these processes at the hardware and software level.
\par
%Along with the definitions of robotics, the applications of it extensively vary which makes it hard to define or understand the task and the responsibilities of robots.
On the other hand, \citeA{lan} and \citeA{lin} list many different real-life robotic applications, from laborer robots to medical-purpose robots and to personal care and companion robots, which contradicts the "Three D's".
While \citeA{pear} also excludes any means of social interaction in the definitions of robotics proposed in this article, \citeA{rodhan} elaborates on this issue by indicating an emerging application of robotics, a molecular level robot which identifies cancerous cells and attacks them by releasing antibodies, thereby eliminating cancerous cells.
Even though experts have not been able to solve dilemmas within the field of robotics, this continuum of sophistication within robotics adds more complex issues to the pile.
% The reason why these social interaction dilemma matter, whether in bodily level or molecular level, is that technology advances in faster pace than that of the wisdom regarding robots and ourselves.
\par
\section{The Most Profound Issues of Robotics}
From its very early conception, AI has been the most controversial issue within the engineering and scientific societies.
The near future is fecund with many dilemmas which humankind, sooner or later, will face regarding AI's moral integrity and its position in society.
This section synthesizes the issues of employment of AI and robots in a wide spectrum of tasks.

\subsection{Safety Issues}
% Since robots can be regarded self-conscience agents, thanks to advanced reasoning, cognitive and perceptual abilities provided by AI, 
Akin to the nature of any given invention, the ultimate reason for the invention of robots is the elevation of life conditions and inquiry about applicable solutions for societal problems.
However, a major issue being discussed is whether or not the military, one of the biggest supporters of robotics, is being sincere with its funding purposes \cite{gre}.
Of the many supporters of such research studies, the military appears to be the most questionable one due to plausible intentions of creation of a humanoid army.
% <readagain>
This utilization of a weaponized robotic army would boost the vast possession of weaponry instead of mediating inter-governmental conflicts.
% </readagain>
This contentious utilization of battlefield robots is one of the biggest concerns when it comes to resolving inter-governmental conflicts.
Such probability may put countries in circumstances strongly resembling the Cold War. \par

% Add human proneness to fail and errors in source code. (skepticism and uncanny)
Coupled with the military concerns, human source code errors are yet another issue impeding roboticists from developing the software.
As computer scientists, fallible human beings, compile millions of lines of code into one "perfect" software, the risk of robot failure due to errors in the code is soaring.
A tiny flaw on the software level, \citeA{lin} note, can lead to fatal results. 

Moreover, hacking has been the most debated issue since the Internet became publicly available.
Along with many other safety issues, hacking is an associated issue regarding the secure integration of robots securely integrating robots in work settings.
In this case, the risk is not identity theft or any other illegal activity. % since the anticipations for robots to be employed in wide variety of tasks.
If a hacker can grant access in a cryptic way to the robots' central processing unit, s/he can make the robot commit any action s/he wants.
So, all of the superior abilities, namely, "its strength, ability to access and operate in difficult environments, expendability and so on", may be turned against society \cite[p.~945]{lin}.
This possibility clearly contradicts the ultimate goal of the creation of robots.

 \subsection{Legal and Ethical Issues}
Furthering concerns about their military applications, robots are envisioned to be controlled by an intelligence which will enable them to perceive their surrounding environments where they can interact with objects by themselves.
Thus, robots can be regarded as independent agents.
This conceptual freedom, however, raises an ethical dilemma caused by the absence of legal ramifications for AI.
The first fatal accident that resulted in a killing of a laborer by a robot is believed to have taken place in the late 70's \cite{lin}.
\citeA{gre} examine that ambiguity by posing a controversial question: "Who will be responsible for the results of AI's actions and failures?"
Since AI and robots have begun replacing laborers in some tasks, this phenomenon indicates that robots will supposedly make their own choices and take actions according to these choices. That controversy causes another interesting question to arise: "Would robots be prosecuted maybe executed because of their failures resulting in fatal accidents?"
Similar to the questions above, there are many considerations that governments need to address by creating regulations and laws with which robots will comply. \par
Furthermore, many researchers have been questioning whether or not these super-intelligent agents should have duties to society, such as paying taxes; if so, to what extent should governments  give social rights to robots?
\citeA{bob} addresses this issue by underlining basic rights and duties of citizens such as paying taxes and voting.
In other words, not only does AI's legal ramification matter, but it is also equally important for governments to decide the AI's responsibility to society, and vice versa.
To address this issue, \citeA{kerb} analyzes Japanese artwork showing futuristic plots based upon robots cohabitating and interacting with humans.
This artwork depicts the very early speculations of social equality between man and machine regarding civil and political rights.


\subsection{Psychological Challenges and Socio-economic Issues}
Social interaction is what differentiates humans from machines.
However, as the research studies toward creation of AI have gone deeper, robots have become more socially interactive.
%For instance, before GPS became publicly available, route planning was based upon map-reading and asking directions of the locals.
%In light of this example, AI has become route planning assistant and has decreased the interrelation among individuals.
%According to \citeA{lan}, reformation of our perception towards AI poses misconceptions regarding personhood and promotes loneliness.
%This issue truly matters due to the fact that alienation from the society relies on that misconception.
%In light of definitions given in the first section, robots have to interact with their physical environment.
According to the definitions of robots in the first section, robots have to interact with their physical environment.
However, such definitions exclude any means of human interactions, even though \citeA{lan} asserts that socially interactive robots are significantly popular in society.
%Albeit popular, social robots, for instance a teacher robot or a guidance robot employed in an art gallery, appears to be decontextualized with regards to definitions provided in the first section.
This discrepancy brings about a major problem on the individual and societal levels.
One example of this fact is that individuals have started to get strong assistance from AI applications, a software crunching numbers faster than a man can do, rather than a close friend who might know better about the individual seeking recommendations, such as his/her taste in music or books.
Even though this issue seems to have minor flip-sides, once they are aggregated \citeA{lan} claims, they will create true alienation from society and confusion regarding personhood. \par
Another matter that deeply affects human psychology is that innovative superiority of robots seems to increase profitability and efficiency in the workplace, thereby undermining the true value of failures.
Given that failures are one of most essential component of the human learning process, the superiority of AI tends to discourage individuals from committing any form of actions due to the fear of failure. 
This fear impedes the same body of individuals to learn, completely hindering them from trusting human judgment.
Depression and other similar psychological challenges are the most profound indicators of the aftermath of an inaccurate definition of AI. \par

Furthermore, the rise of computers has led to the lessening of human employment starting from the late 70's.
Because of the downsizing tendency of corporations which has resulted from this replacement, millions of low-skilled workers have lost their jobs.
Robots and AI supposedly represent the maxima of accuracy and repeatability, which promotes reliance on AI, thanks to their cognitive and learning abilities that outperform human abilities.
Given that historic fact, \citeA{gre} predict that this reliance might cause an inevitable replacement of low-skilled laborers with the more powerful robots, akin to the aftermath of the computerization era.
In light of that prediction, highly developed economies may show a strong recession due to the significant loss in human work power and if this aggressive expansion continues, other occupations and professions will be under the same risk of mechanization. \par

Along with the economic effects of workplace computerization, eradication of low-skilled jobs may lead to the expansion of the gap among the social groups \cite{gre}.
For instance, as the first paragraph of this section indicates, those who can use computers or know how to develop software have reached the higher classes in societies, while others could not seize the same momentum.
This computerization instructs young individuals belonging to the lower layers of society to believe that there is no way to elevate in society.
This expansion in the gap within society and among the social groups clearly contradicts the societal goals of robotics.

\section{Conclusions}
The notion of increasingly accelerating technology with the valuable help of scientific research studies thrills individuals.

Yet, it is crucially important to decide whether or not robots are simply functional machines that are utilitarian in their responsibilities and nothing more.
If so, it is irrelevant to discuss the ethical problems that have arisen by its creation because the problems will become a question of accuracy of the source code that runs robots.
In that case, the matter that should be discussed would be how to mitigate, even eradicate, human errors taking place when roboticists develop robots and AI. \par
As for the problems, there are two distinctive approaches to the future of robots. The first one is that robots will highly likely be no more prone to errors than any other human invention.
If evolutionary psychology is taken into account, the first fully autonomous robots will also evolve rapidly toward their excellence.
Moreover, the second approach claims that as the expansion and the accessibility to AI and robotics increase, the prediction of the aftermath has become obsolete due to the late response of individuals and their wisdom toward these technologies.
Thus, robotics should be thoroughly scrutinized by various roboticists, scientists, ethicists, psychologists, sociologists and many other experts to understand and formulate its limitations and endless possibilities. Otherwise, yet another issue will develop into controversy before former ones can be solved like in the case of genetically modified organisms and the Internet.  
\bibliographystyle{apacite}
\bibliography{bibliography}
\end{document}