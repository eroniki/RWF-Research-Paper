% plagiarism check
% cohesion check
% grammar check (firt two page of new paper)
\documentclass[man]{apa6}
\title{Robots, Are They Overlords or Serf-Laborers?}
\author{Murat Ambarkutuk}
\affiliation{English Language Institute, University of Delaware \\ murata@udel.
edu}
\note{First Draft; 6/8/2015; \LaTeX}
\usepackage{apacite}
% modify it!
\abstract{Of all the tools and methods invented by humankind to tame nature and make use of its challenges, the invention of computer has had the greatest impact on society.
Coupled with the Internet, computers have changed the way billions of humans work, entertain themselves, and think.
Since the rise of the computers, Artificial Intelligence (AI) has been one of the most debated issues in the field of engineering.
Robots are also subject to these debates because that tangible spin-off technology that will be run by such an AI that will have superior cognitive and perceptive abilities than that of man. 
Many researchers argue that the creation of AI will deeply elevate society and it will help humankind to solve major societal problems, while others refute the arguments for that conceptual invention.
This paper explores these issues that the creation of AI will raise, and it attributes these problems to widely varying definitions and the applications of these technologies.
}
\keywords{artificial intelligence, robotics, ethics, society} 
\shorttitle{Overlords vs. Serf-Laborers}
\leftheader{Even-Numbered Page Header} %Change it
\begin{document}
\maketitle

\section{Introduction}
Even though the term robot was coined two centuries ago and preliminary contemplation of Artificial Intelligence (AI) emerged decades ago, some of the problems that have been raised by these two terms are yet to be solved.
Sometimes, solutions for any arbitrary problem can be addressed by seeing the bigger picture rather than scrutinizing tiny details.
Thus, this paper explores these problems and then analyzes the numerous definitions and applications of these technologies which pose challenges on varying levels, from individual to societal.
The formation of this paper is as follows: The first section categorizes the issues that robotics have arisen; the second and third sections then scrutinize the definitions and the applications of robots, respectively, to attribute the reasons of these issues. 

\section{The Most Profound Issues of Robotics}
From the very beginning of the conceptual designs of AI, it has been the most debated issue in the engineering and scientific societies due to its ethical problems.
Near future is fecund to a dilemma which humankind will face in AI's moral integrity and its position in society.
This section synthesizes the issues risen by the creation of AI and employment robots in a wide spectrum of tasks.

\subsection{Safety Issues}
% Since robots can be regarded self-conscience agents, thanks to advanced reasoning, cognitive and perceptual abilities provided by AI, 
A major issue having been discussed is whether or not military, one of the biggest supporter of robotics, is being sincere with its funding purposes \cite{gre}.
Of the many supporters of such research studies, military appears to be the most questionable one due to plausible intentions of creation of humanoid army.
The ultimate reason of robots which is elevation of life conditions and inquiry for applicable solutions for the chronic problems of our societies, by the nature of any given tools that invented by humankind, notwithstanding, utilization of weaponized robotic army would increment the vast possession of ammunition and weaponization problem rather than solving chronic problems of relationships among governments.
The probable mass employment of battlefield robots is one of the biggest concern when it comes to mediate a inter-governmental conflict. Such probability may put countries in circumstances which closely resembles to the Cold War. \par

 
\subsection{Legal and Ethical Issues}
Robots are envisioned to be controlled by a such intelligence which enable their to perceive their surrounding environments where they interact with the objects. Thus, they can be regarded as independent agents.
This conceptual independency, however, raises an ethical dilemma caused by the absence of legal ramification of AI.
Many researchers has been questioning whether or not these super-intelligent agents should have duties to the society and to what extent governments give social rights \cite{bob}.
Given that approach, \cite{kerb} mentions early speculations of social equality regarding civil and political rights among robots and humans in Japanese anime and mangas which artworks are %consisted of futuristic plots including robots cohabitating and interacting with humans.
\citeA{gre} discusses that it is not obvious that who will be responsible for results of AI's actions.
Not only does AI's position in the society matter, but it is also equally important for governments to decide the AI's responsibility for the society, vice versa.

\subsection{Socio-economic Issues}
The rise of computers which has led the employment of robots in industry lessen the human employment starting from late 70's.
Because of the downsizing tendency of corporations which resulted from this replacement, millions of low-skilled workers lost their jobs.
Given that fact, \citeA{gre} predicts the aftermath of the creation of AI.
This vast computation power might replace most of the low-skilled labourers with the more powerful robots.
In the light of that prediction, highly developed economies may show an extensive recession. \par
Along with the economical effects of workplace computerization, eradication of low-skilled jobs may lead the expansion of the gap among the social groups.
For instance, those who can use computers or know develop software have reached to higher classes in the societies.
This expansion of gap among the social groups contradicts the societal goals of the creation of AI. \par
In addition to former arguments, \citeA{rosa} proposes yet another.
Even though researchers, for instance, want to create a narrow-domain savvy robot, the robot may want to change its profession from the one that it had been taught or coded to a brand new one.
This probable change of profession contradicts with the reason its creation purpose which may be interpreted as a act of disobey to its master.

\section{The Bigger Picture}
\subsection{What is a Robot?}
%One common definition of AI is cited by \citeA{bost}, "A super-intelligence is any intellect that vastly outperforms the best human brains in practically every field, including scientific creativity, general %wisdom and social skills."
%This definition clearly distinguishes specific domain intelligent agents and AI.
% Based on this definition a computer system, Deep Blue, which is a well-known computer system for its success in chess can be considered smart within a narrow domain rather than a genuine AI supported by limitations that \citeauthor{duff} \citeyear[p.~34]{duff} speculates.
\par

By literary definition the term robot is derived from the Czech word "roboti" meaning serf-labor. 
However, this definition closely referring to slave-like labourer has been altered to something brand new as technology advances.
In his article entitled "The Sheer Difficulty of Defining What a Robot Is", \citeA{pear} discusses two distinctive definitions of robotics that are posed by the prominent experts in the field of robotics.
These two definitions emphasize different aspects of robotics creating important distinctions among them.
The first definition puts emphasis on the successful combination of a rigid body responsible for interacting with its environment and a central reasoning unit responsible for reasoning tasks based perception and cognition.
Without a software and hardware collaboration, as \citeA{pear} cites in his article, accomplishing its tasks for a robot cannot be possible.

The second definition, however, takes autonomy into account.
A robot, \citeA{pear} quotes in his article, must have an detailed idea of any action formed by a tedious plan, accurate reasoning and precise action to interact with its physical environment which closely aligns with the definition provided by \cite{lin}. 
\citeauthor{lin} \citeyear[p.~943]{lin} define a robots as "an engineered machine that senses, thinks, and acts".
In light of these definitions, robots have to interact with its environment, which exclude any means of social interactions.
However, \citeA{lan} asserts that socially interaction robots are significantly popular in society.

%Albeit popular, social robots, for instance a teacher robot or a guidance robot employed in an art gallery, seem to be decontextualized.

This discrepancy brings about a major problem at the individual and societal levels.
One example of this fact is that individuals have started to get strong advice from AI applications, a software crunching numbers faster than a man can do, rather than a close friend who might know better about the individual seeking advice, such as the taste in music or book.
Even though this issue seems to have a minor flipside, once it is aggregated \citeA{lan} claims, it will create true alienation from society and confusion regarding personhood.
In other words, roboticists have not been able  to reach a consensus on the definition of robots the embodiment of AI, \citeA{lin} noted.
\par

\subsection{What (should) does a Robot do?}
% A strip from the old paper
In order to make its obfuscated definition clear, the tasks of AI should be addressed.
On the one hand,  \citeA[p.~943]{lin} note that robots are supposedly employed in such tasks called "three D's" referring "dull, dirty, or dangerous" tasks.
From that instruction, a robot should replace human where the task contains either mundane chores, or risks of contamination, or threatening factors for human lives.
\par 
%The task of AI is to create methods to enable machines to obtain symbolic data sets and manipulate them in order to solve generic problems in the most energy efficient and fastest fashion.
%The theory suggests that it is possible to achieve by imitating the processes of logical reasoning and decision making.
%Consequently, cognitive and perceptual systems of the human body have been scrutinized by researchers in order to emulate these processes at the hardware and software level.
\par
%Along with the definitions of robotics, the applications of it extensively vary which makes it hard to define or understand the task and the responsibilities of robots.
On the other hand, \citeA{lan} and \citeA{lin} list many different real life robotic applications, from labourer robots to medical purpose robots and to personal care and companion robots, which contradicts with given instruction.
While \citeA{pear} also excludes any means of social interaction in the definitions of robotics proposed in this article, \citeA{rodhan} elaborates this issue by indicating an emerging application of robotics, a molecular level robot which identifies cancerous cells and attack them by releasing antibodies, thereby eliminating cancerous cells.
Even though experts have not been able to solve dilemmas within the field of robotics, this continuum of sophistication within robotics add more complex issues to the pile.
% The reason why these social interaction dilemma matter, whether in bodily level or molecular level, is that technology advances in faster pace than that of the wisdom regarding robots and ourselves.
\par

\section{Conclusion}
The notion of increasingly accelerating technology with the precious help of scientific research studies thrills individuals, and gives hope to patients.
%However, these two prominent concepts of technology closely resembles another, the Internet.
%, otherwise yet another issue will jump into debate topics before formers could be solved.
Yet, it is crucially important to decide whether or not robots are simply functional machines that are utilitarian in their responsibilities and nothing more.
If so, it is irrelevant to discuss the ethical problems arisen by its creation because the problems will become a question of accuracy of the source code that runs robots and forms AI.
In that case, the matter should be discussed would be how to prepare individuals, societies and governments its coming. \par
As for the ethical problems, they will highly likely to be no more prone to erroneousness than any other human invention.
If evolutionary and development psychology is taken into account, the first members of fully autonomous social robots will also evolve rapidly to their excellence.
As the expansion and the accessibility to AI and robotics increase, the prediction of the aftermath has become decrepit due to the late response of individuals and their wisdom toward these technologies.
Thus, robotics and AI should be scrutinized to understand and formulate its definitions and functions, otherwise yet another issue will jump into debate topics before formers could be solved.
\bibliographystyle{apacite}
\bibliography{bibliography}
\end{document}