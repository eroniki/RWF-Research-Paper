% plagiarism check
% cohesion check
% grammar check (firt two page of new paper)
\documentclass[man]{apa6}
\title{Robots, Are They Overlords or Serf-Laborers?}
\author{Murat Ambarkutuk}
\affiliation{English Language Institute, University of Delaware \\ murata@udel.edu}
\note{First Draft; 6/8/2015; \LaTeX}
\usepackage{apacite}
% modify it!
\abstract{Of all the tools and methods invented by humankind to tame nature and make use of its challenges, the invention of computer has had the greatest impact on society. Coupled with the Internet, computers have changed the way billions of humans work, entertain themselves, and think. Since the rise of the computers, Artificial Intelligence (AI) has been one of the most debated issues in the field of engineering. Research toward the creation of AI includes a creation of self-aware machines that have superior intellectual abilities to that of human abilities. However, many researchers argue that the creation of AI will deeply change the society and it will help humankind to solve major societal problems, while others refute the arguments for the conceptual invention because of ethical problems that AI will bring such as an anticipated human annihilation by these agents. This paper focuses on the the ethical problems that the creation of AI will raise.} 
\keywords{artificial intelligence, robotics, ethics, society} 
\shorttitle{Overlords vs. Serf-Laborers}
\leftheader{Even-Numbered Page Header} %Change it
\begin{document}
\maketitle
% Citing examples and the author lists
% \citeA{bost} \citeA{bry} \citeA{bob} \citeA{gre} \citeA{rosa} \citeA{duff} \citeA{lin}
\section{Introduction}
Even though the term robot was coined two centuries ago and the idea of AI emerged decades ago, some of the problems that have been raised by these two terms are yet to be solved. Sometimes, solutions for any arbitrary problem can be solved by seeing the bigger picture rather than scrutinizing tiny details. Thus, this paper focuses on numerous definitions for these terms and various applications of them which create intricacies and challenges in many level, varying from individual to societal level. Formation of this paper is as it follows: \#\#, \#\#, \#\#

\section{The Bigger Picture}
\subsection{What is a Robot?}
One common definition of AI is cited by \citeA{bost}, "A super-intelligence is any intellect that is vastly outperforms the best human brains in practically every field, including scientific creativity, general wisdom and social skills." This definition clearly distinguishes specific domain intelligent agents and AI. Based on this definition a computer system, Deep Blue, which is a well-known computer system for its success in chess can be considered smart within a narrow domain rather than a genuine AI supported by limitations that \citeauthor{duff} \citeyear[p.~34]{duff} speculates.\par

By literary definition the term robot is derived from the Czech word "roboti" meaning serf-labor. However, this definition closely refering to slave-like labourer has altered to something brand new as technology advances. In his article entitled "The Sheer Difficulty of Defining What a Robot Is", \citeA{pear} discusses two distinctive definitions of robotics that are posed by the prominent experts in the field of robotics. These two definitions emphasize different aspect of robotics creating important distinctions among them. The first definition puts emphasis on the successful combination of a rigid body responsible for interacting with its environment and a central reasoning unit responsible for reasoning tasks based perception and cognition. Without a software and hardware collaboration, as \citeA{pear} cited in his article, accomplishing its tasks for a robot cannot be possible.
The second definition, however, takes autonomy into account. A robot, \citeA{pear} quoted in his article, must have an detailed idea of any action formed by a tedious plan, accurate reasoning and precise action to interact with its physical environment. In light of these definitions, robots have to interact with its environment, which exclude any means of social interactions. However, \citeA{lan} asserts that socially interaction robots are significantly popular among society. Albeit popular, social robots, for instance a teacher robot or a guidance robot employed in an art gallery, seem to be decontextualized. This discrepancy brings about a major problem in individual and societal level. One example of this fact is that individuals have started to get strong advice from AI applications, a software crunching numbers faster than a man can do, rather than a close friend who might know better about the individual seeking advice, such as the taste in music or book. Even though this issue seems to have a minor flipside, once it is aggregated \citeA{lan} claims, it will create true alienation from society and confusion regarding personhood. In other words, roboticists have not been able  to reach a consensus on the definition of robots the embodiment of AI, \citeA{lin} noted.\par
\subsection{What (should) does a Robot do?}
% A strip from the old paper
In order to make clear its obfuscated definition, the tasks of AI should be addressed. The task of AI is to create methods to enable machines to obtain symbolic data sets and manipulate them in order to solve generic problems in the most energy efficient and fastest fashion. The theory suggests that it is possible to achieve by imitating the processes of logical reasoning and decision making. Consequently, cognitive and perceptual systems of the human body have been scrutinized by researchers in order to emulate these processes at the hardware and software level.\par
Along with the definitions of robotics, the applications of it extensively vary which makes it hard to define or understand the task and the responsibilities of robots. \citeA{lan} and \citeA{lin} listed many different applications of robotics, from labourer robots to medical purpose robots and to personal care and companion robots.
Although \citeA{pear} excludes any means of social interaction in the definitions of robotics proposed in this article, \citeA{rodhan} exemplifies an emerging application of robotics, a molecular level robot which identifies cancerous cells and attack them by releasing antibodies, thereby eliminating cancerous cells. Even though, experts have not been able to solve dilemmas within the field of robotics, this steadily growing sophistication within robotics add more complex issues to the pile.
% The reason why these social interaction dilemma matter, whether in bodily level or molecular level, is that technology advances in faster pace than that of the wisdom regarding robots and ourselves.\par

\section{Conclusion}
The notion of increasingly accelerating technology with the precious help of scientific research studies thrills individuals, and gives hope to patients.
%However, these two prominent terms of technology closely resembles another, the Internet.
%, otherwise yet another issue will jump into debate topics before formers could be solved. 
Yet, it is crucially important to decide whether or not robots are simply functional machines that are utilitarian in their responsibilities and nothing more. If so, it is irrelevant to discuss the ethical problems arisen by its creation because the problems will become a question of accuracy of the source code that runs robots and forms AI. In that case, the matter should be discussed would be how to prepare individuals, societies and governments its coming. As for the ethical problems, they will highly likely to be no more prone to erroneousness than any other human invention. If evolutionary and development psychology is taken into account, the first members of fully autonomous social robots will also evolve rapidly to their excellence. As the expansion and the accessibility to AI and robotics increase, the prediction of the aftermath has become decrepit due to the late response of individuals and their wisdom toward these technologies. Thus, robotics and AI should be scrutinized to understand and formulate its definitions and functions, otherwise yet another issue will jump into debate topics before formers could be solved. 
\bibliographystyle{apacite}
\bibliography{bibliography}
\end{document}
%\section{Advantages of Artificial Intelligence}
%Anthropomorphism is the basis of AI which is the imitation of the way that humans find solutions for problems. As a result of this imitation process, anticipated outcomes are not only solving the problem itself, but also finding the optimal solution for the problem in shortest time period possible. This speed and ability to solve problems will be advantageous for humankind. In this section, the advantages of AI are discussed. 
%\subsection{Solutions for the Chronic Problems of Societies}
%\citeA{bry} claims that the common societal problems, such as the inequity of income distribution and poverty resulting from overpopulation, are impossible to cope with using regular tools and frameworks which governments now have. Having been scrutinizing social sciences, namely, developmental psychology and sociology, researchers might expand their interpretation of human behaviors which might enable them find new solutions for the given problems. Moreover, elimination of aging and eradication of diseases might be the possible consequence of the creation of AI \citeA{bost}. Both authors discuss AI's advanced reasoning and interpretation underlining the new opportunities that AI might bring to societies in order to solve chronic societal problems.
%\subsection{The Last Invention of Humankind That Researchers Forsee}
%By its definition, AI appears to provide the fastest and the optimal solutions for the problems that is given which depends on supreme decision-making and cognitive processes that of humans. Consequently, this advanced techniques will likely to provide a privilege to have our problems solved with the help of AI. In other words, thanks to outstripping abilities of AI, humans might not need to invent any other tool to solve problems \cite{bost}.
%\subsection{Singularity}
%\citeA{bost} also discusses that AI might deliberately elaborate its structure and source code to enhance its abilities resulting in more advanced superintelligence, which will result in more complex structure of AI and calculation power.
%\subsection{Availability of Cloning}
%Since AI will notably rely on software implementations, it might be convenient to download its source code and promptly mount it on new hardware. That simplicity will likely to create more calculation power for the society and enable individuals to have personal robots. This possibility to create personal superintelligent agents with a cost of fractions of seconds and hardware.
%\section{Challenges Raised by AI}
%From the very beginning of the conceptual designs of AI, it has been the most debated issue in the engineering and scientific societies due to its ethical problems. Near future is fecund to a dilemma which humankind will face in AI's moral integrity and its position in society \cite{duff}. This section synthesizes the ethical issues and counterarguments coping with them.
%\subsection{Security Issues}
%Since robots can be regarded self-conscience agents, thanks to advanced reasoning, cognitive and perceptual abilities provided by AI, a major issue having been discussed is whether or not military, one of the biggest supporter of AI, is being sincere with its funding \cite{gre}. Of the many supporters of AI, military appears to be the most questionable one due to plausible intentions of creation of humanoid army. The ultimate reason of AI which is elevation of life conditions and inquiry for the chronic problems of our societies notwithstanding, utilization of weaponized robotic army would increment the vast possession of ammunition and weaponization problem rather than solving our problems.\par
%Although being able to elaborate its source code by itself and being easily cloned are the factors considered advantage for the society, \citeA{duff} claims that combination of these factors would make AI able to easily reproduce itself, which means robots may own robots that they create in the long term. Consequently, second generations robots created by the first generations would be utterly dependent on the first generation robots which means the first generation robots might command their offspring how ever they want. This unpredictable action-decision chain raises a major security problems including human annihilation for the society. 
%\subsection{Socio-economic Issues}
%The rise of computers which has led the employment of robots in industry lessen the human employment starting from late 70's. Because of the downsizing tendency of corporation which resulted from this replacement, millions of low-skilled workers lost their jobs. Given that fact, \citeA{gre} predicts the aftermath of the creation of AI. This vast computation power might replace most of the low-skilled laborers with the more powerful robots. In the light of that prediction, highly developed economies may show an extensive recession.\par
%Along with the economical effects of workplace computerization, eradication of low-skilled jobs may lead the expansion of the gap among the social groups. For instance, those who can use computers or know develop software have reached to higher classes in the societies. This expansion of gap among the social groups contradicts the societal goals of the creation of AI.\par
%In addition to former arguments, \citeA{rosa} proposes yet another. Even though researchers, for instance, want to create a narrow-domain savvy robot, the robot may want to change its profession from the one that it had been taught or coded to a brand new one. This probable change of profession contradicts with the reason its creation purpose which may be interpreted as a act of disobey to its master.
%\subsection{Legal Issues}
%Since self-awareness of AI, it can be regarded as an independent agent. This conceptual independency, however, raises a ethical dilemma. Many researchers has been questioning whether or not superintelligent agents have duties to the society and to what extent governments give social rights \cite{bob}. Given that approach, \cite{kerb} mentions early speculations of social equality regarding civil and political rights among robots and humans in Japanese anime and mangas which artworks are consisted of futuristic plots including robots cohabitating and interacting with humans. These questions caused by the legal ramification of AI. \citeA{gre} discusses that it is not obvious that who will be responsible for results of AI's actions. Not only does AI's position in the society matter, but it is also equally important for governments to decide the AI's responsibility for the society, vice versa. \par
%The counterargument was proposed by \citeA{rosa} by underlining the evolutionary approach on human morality. As \citeA{rosa} cited, probability of error in evolution of human morality and nature displays a significant resemblance to the errors that might AI commit. For that reason, it is more logical to create a ground that can tolerate the probable AI's errors.
%\subsection{Religious Issues}
%One issue that discussed among scholars is whether or not creating a self-aware machine which is able to reason in the same level with human intelligence would be stealing the role of God. In addition to that issue, almost all of the monotheistic religions put significant emphasis on the features of God concept. According to these religions, God himself is unique and incomparably exceptional than the ones what he created. Of the many major sins, imitating God, or in other words one's God-like manifestation, is the unforgivable one.\citeA{gre} discusses this issue by showing the similarities of creation of a self-aware agent and creation concept of God. Moreover, \citeA{duff} mentions human-like form that researchers tend to build social robots, an example of embodiment of AI. In the light of these facts, the concept of God and the creation of AI seems to intersect to some extent.\par
%The counterarguments put significance on more God and abilities that humans given by him rather than the limitations provided by the rules of religions. This approach to research and development of AI considers human and nature the masterpiece of God. For instance, brain is one the biggest resource that provided by God in order to make humans able to create tools and techniques to solve their problems, namely, taming the nature, building societies and elevating one's position in these societies. From this perspective, AI would be the best use of brain to commit what God orders \cite{gre}.